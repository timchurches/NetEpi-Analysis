%
%   The contents of this file are subject to the HACOS License Version 1.2
%   (the "License"); you may not use this file except in compliance with
%   the License.  Software distributed under the License is distributed
%   on an "AS IS" basis, WITHOUT WARRANTY OF ANY KIND, either express or
%   implied. See the LICENSE file for the specific language governing
%   rights and limitations under the License.  The Original Software
%   is "NetEpi Analysis". The Initial Developer of the Original
%   Software is the Health Administration Corporation, incorporated in
%   the State of New South Wales, Australia.
%
%   Copyright (C) 2004,2005 Health Administration Corporation.
%   All Rights Reserved.
%
\section{\module{soomarray}}

\declaremodule{standard}{soomarray}

The \module{soomarray} module contains a Python wrapper around the
\module{blobstore} extension module.

The \module{soomarray} module contains the following:

\begin{excdesc}{Error}
Exception raised when unexpected data is located in the BLOB store.
\end{excdesc}

\begin{classdesc}{MmapArray}{blob}
Return a new instance of the \class{MmapArray} class using the
\class{Blob} object specified in the \var{blob} argument.
\end{classdesc}

\begin{classdesc}{ArrayDict}{filename, mode \code{= 'r'}}
Return a new instance of the \class{ArrayDict} class.

The \var{filename} argument specifies the name of the file which will
be used to store and retrieve memory arrays.  The \var{mode} argument
is can be \code{'r'} to access an existing file in read only mode,
\code{'r+'} to access an existing file in read-write mode, or
\code{'w+'} to create a new file in read-write mode.
\end{classdesc}

\subsection{ArrayDict Objects}

These objects behave like a Python dictionary.  Each item stored must
be either a plain Numpy array, or a masked array from the MA package.

Internally the object manages a \class{BlobStore} object in which the
BLOB at index zero contains a pickled dictionary which yields the
index of the BLOB which contains the Numpy array.  For masked arrays,
the BLOB referenced by the dictionary contains the index of the mask
array in the \member{other} BLOB member.

You must be careful when modifying the contents of the
\class{ArrayDict} object because the internal \class{BlobStore} may
need to grow the memory mapped file.  Growing the file will invalidate
all of the addresses contained in arrays retrieved from the file.  The
\class{MmapArray} objects are able to automatically handle this event,
but any arrays derived from these will not.  If you slice and dice
\class{MmapArray} objects then modify their containing
\class{ArrayDict}, you will get a segmentation fault.

\begin{verbatim}
>>> from Numeric import *
>>> from soomarray import ArrayDict 
>>> ad = ArrayDict('file.dat', 'w+')
>>> ad['a'] = array(xrange(100))
>>> ad.keys()
['a']
>>> len(ad) 
1
>>> ad['b'] = array(range(10))
>>> ad['b']
[0,1,2,3,4,5,6,7,8,9,]
>>> from soomfunc import *
>>> intersect(ad['a'], ad['b']) 
[0,1,2,3,4,5,6,7,8,9,]
\end{verbatim}

\class{ArrayDict} objects have the following interface:

\begin{methoddesc}[ArrayDict]{__del__}{}
When the last reference to the object is dropped, the object will
repickle the internal array dictionary to BLOB zero in the associated
\class{BlobStore} object.
\end{methoddesc}

\begin{methoddesc}[ArrayDict]{__getitem__}{key}
Returns a \class{MmapArray} object indexed by the \var{key} argument.
This is called to evaluate \code{ad[key]}.  Raises
\exception{KeyError} if \var{key} is not a key in the dictionary.
\end{methoddesc}

\begin{methoddesc}[ArrayDict]{__setitem__}{key, a}
Called to implement assignment of \var{a} to \code{ad[key]}.
\end{methoddesc}

\begin{methoddesc}[ArrayDict]{__delitem__}{key}
Called to delete \code{ad[key]}.  Raises \exception{KeyError} if
\var{key} is not a key in the dictionary.
\end{methoddesc}

\begin{methoddesc}[ArrayDict]{__len__}{}
Returns the number of items stored in the dictionary.  A masked array
counts as one item.
\end{methoddesc}

\begin{methoddesc}[ArrayDict]{clear}{}
Deletes all contents.
\end{methoddesc}

\begin{methoddesc}[ArrayDict]{get}{key \optional{, default \code{= None}}}
Implements the dictionary \method{get()} method.
\end{methoddesc}

\begin{methoddesc}[ArrayDict]{has_key}{key}
Returns TRUE if \var{key} is a valid key in the dictionary.
\end{methoddesc}

\begin{methoddesc}[ArrayDict]{keys}{}
Returns a list containing all of the keys in the dictionary.
\end{methoddesc}

\begin{methoddesc}[ArrayDict]{values}{}
Returns a list containing all of the arrays in the dictionary.
\end{methoddesc}

\begin{methoddesc}[ArrayDict]{items}{key}
Returns a list of tuples containing all \code{(key, array)} pairs in
the dictionary.
\end{methoddesc}

\subsection{MmapArray Objects}

Implements a Numpy array which has memory mapped data within a
\class{BlobStore}.  These objects act exactly like the \class{UserArray}
object from the Numpy package (they should since their code was pasted
from the Numpy source).

Internally they wrap a \class{Blob} object which was retrieved from
the parent \class{BlobStore} object.  Each time the array is used the
object calls the \class{Blob} \method{as_array()} method to refresh
the address of the array data.

Apart from the standard Numpy array methods the \class{MmapArray}
class implements the following:

\begin{methoddesc}[MmapArray]{append}{a}
This method appends the array in the \var{a} argument to the BLOB.
The \class{BlobStore} container will resize as required.  This is a
handy way to build up an array which is too large to fit in memory.

\begin{verbatim}
>>> from Numeric import *
>>> from soomarray import ArrayDict 
>>> ad = ArrayDict('file.dat', 'w+')
>>> ad['a'] = array(xrange(100))
>>> a = ad['a']
>>> a.append(xrange(100))
>>> len(a) 
200
\end{verbatim}
\end{methoddesc}
