%
%   The contents of this file are subject to the HACOS License Version 1.2
%   (the "License"); you may not use this file except in compliance with
%   the License.  Software distributed under the License is distributed
%   on an "AS IS" basis, WITHOUT WARRANTY OF ANY KIND, either express or
%   implied. See the LICENSE file for the specific language governing
%   rights and limitations under the License.  The Original Software
%   is "NetEpi Analysis". The Initial Developer of the Original
%   Software is the Health Administration Corporation, incorporated in
%   the State of New South Wales, Australia.
%
%   Copyright (C) 2004,2005 Health Administration Corporation.
%   All Rights Reserved.
%
\section{\module{blobstore}}

\declaremodule{extension}{blobstore}

The \module{blobstore} extension module contains a fairly thin wrapper
on top of the memory mapped BLOB storage which is implemented in the
Storage API.

The \module{blobstore} module contains the following:

\begin{datadesc}{__version__}
A string which specifies the version of the \module{blobstore} module.
\end{datadesc}

\begin{funcdesc}{open}{filename \optional{, mode \code{= 'r'}}}
Open the file named in the \var{filename} argument using the specified
\var{mode}.  Returns a \class{BlobStore} object.

\begin{verbatim}
>>> import blobstore
>>> bs = blobstore.open('file.dat')
\end{verbatim}
\end{funcdesc}

\subsection{BlobStore Objects}

These objects are used to manage a sequence of BLOB items.  The object
behaves much like a Python list.  Each element in the list is a
\class{Blob} object.

\class{BlobStore} objects have the following interface:

\begin{methoddesc}[BlobStore]{__len__}{}
Returns the number of blobs in the blob store.
\end{methoddesc}

\begin{methoddesc}[BlobStore]{__getitem__}{i}
Uses the blob store sequence to retrieve the blob descriptor indexed
by \var{i} and returns it as a new instance of the \class{Blob} class.
The blob data is accessed via the \class{Blob} object.
\end{methoddesc}

\begin{methoddesc}[BlobStore]{append}{}
Reserves an index for a new blob in the blob store.  The reserved
index is returned.  Note that no data will be allocated in the blob
store until data is saved to that blob index.
\end{methoddesc}

\begin{methoddesc}[BlobStore]{get}{i}
Retrieves the raw blob descriptor indexed by \var{i} and returns it as
a new instance of the \class{Blob} class.  This by-passes the blob
store sequence and indexes directly into the blob table.  This allows
you to see the blobs which are used to store the blob sequence and
table as well as free blobs.  The \class{Blob} objects obtained this
way cannot be used to load or save data.
\end{methoddesc}

\begin{methoddesc}[BlobStore]{free}{i}
Free the storage associated with the blob in the sequence at index
\var{i}.
\end{methoddesc}

\begin{methoddesc}[BlobStore]{usage}{}
Returns a tuple which contains two numbers; the amount of data storage
allocated in the blob store, and the amount of space free or wasted in
the blob store.
\end{methoddesc}

\begin{methoddesc}[BlobStore]{header}{}
Returns a tuple containing the following values;
\begin{longtable}{l|l}
table_size  & Number of blob table entries allocated. \\
table_len   & Number of blob table entries in use. \\
table_index & Table index of the blob which contains the blob table. \\
table_loc   & File offset of the start of the blob table. \\
seq_size    & Number of blob sequence entries allocated. \\
seq_len	    & Length of the blob sequence. \\
seq_index   & Table index of the blob which contains the blob sequence. \\
\end{longtable}

To understand these numbers you should refer to the documentation of
the Storage API.
\end{methoddesc}

\subsection{Blob Objects}

Blobs retrieved from a \class{BlobStore} object are returned as
\class{Blob} objects.  The \class{Blob} objects provide access to the
data associated with a blob in a blob store.  The \class{Blob} object
retains a reference to the containing \class{BlobStore} and they also
store their index within the blob store.

\class{Blob} objects have the following interface:

\begin{memberdesc}[Blob]{len}
This read-only member contains the length of the blob data.
\end{memberdesc}

\begin{memberdesc}[Blob]{size}
This read-only member contains the allocated size of the blob.  The
blob \member{size} will always be at least as large as the blob
\member{len}.  Blob allocations are aligned to the size of an integer.

When a blob is resized or freed it can leave some free space in the
blob store.  When allocating a new blob, the smallest free area which
can contain the new data will be used in preference to growing the
file.
\end{memberdesc}

\begin{memberdesc}[Blob]{loc}
This read-only member contains the offset within the file of the blob
data.
\end{memberdesc}

\begin{memberdesc}[Blob]{status}
This read-only member contains the status of the blob.  The values and
their meaning are:

\begin{longtable}{l|l}
0 & The blob is free space. \\
1 & The blob contains the blob table. \\
2 & The blob contains the blob sequence. \\
3 & The blob contains user data. \\
\end{longtable}
\end{memberdesc}

\begin{memberdesc}[Blob]{other}
This integer member can be used by your code.  The \class{ArrayDict}
class uses it to link the mask and data parts of a masked array.
\end{memberdesc}

\begin{memberdesc}[Blob]{type}
This integer member can be used by your code.
\end{memberdesc}

\begin{methoddesc}[Blob]{as_array}{}
Returns a Numpy array object which uses the memory mapped blob data as
the contents of the array.  The array object is cached internally, so
you can call this without penalty.

Every time you call this method the cached Numpy array will have its
internal data pointer updated to protect against blob store resizing.
\end{methoddesc}

\begin{methoddesc}[Blob]{as_str}{}
Returns a Python string which is constructed using the blob data.  The
string allocates its own copy of the data.  The string is cached
internally, so you can call this without penalty.
\end{methoddesc}

\begin{methoddesc}[Blob]{save_array}{a}
Save the array passed in the \var{a} argument in the blob associated
with this object.
\end{methoddesc}

\begin{methoddesc}[Blob]{append_array}{a}
Append the array passed in the \var{a} argument to the blob associated
with this object.
\end{methoddesc}

\begin{methoddesc}[Blob]{save_str}{s}
Save the string passed in the \var{s} argument in the blob associated
with this object.
\end{methoddesc}
