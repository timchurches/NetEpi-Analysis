%
%   The contents of this file are subject to the HACOS License Version 1.2
%   (the "License"); you may not use this file except in compliance with
%   the License.  Software distributed under the License is distributed
%   on an "AS IS" basis, WITHOUT WARRANTY OF ANY KIND, either express or
%   implied. See the LICENSE file for the specific language governing
%   rights and limitations under the License.  The Original Software
%   is "NetEpi Analysis". The Initial Developer of the Original
%   Software is the Health Administration Corporation, incorporated in
%   the State of New South Wales, Australia.
%
%   Copyright (C) 2004,2005 Health Administration Corporation.
%   All Rights Reserved.
%
\section{Storage API}

The \module{blobstore} extension module is built on top of the Storage
API.  This API provides a blob storage via a memory mapped file.

The memory mapped file managed by the Storage API is logically
arranged like the following figure:

\begin{figure}[h]
\begin{center}
\includegraphics{storage}
\caption{Memory Mapped File Layout}
\end{center}
\end{figure}

There is a simplicifcation in the above diagram.  The blob containing
the blob sequence (labelled ``seq'') contains blob numbers which are
then used as indexes back into the blob table to locate the actual
blob data.

At offset zero in the file is the file header which is a
\ctype{StoreHeader} structure from which everything else is located.

\begin{ctypedesc}[StoreHeader]{StoreHeader}
This structure which is contained at offset zero in the memory mapped
file.  Its definition, found in \file{storage.h}, is:

\begin{verbatim}
typedef struct {
    off_t table_loc;            /* where the blob table is stored */
    int table_size;             /* allocated table size */
    int table_len;              /* entries used in the table */
    int table_index;            /* blob which contains the blob table */

    int seq_index;              /* blob which contains the array lookup */
    int seq_size;               /* allocated sequence size */
    int seq_len;                /* number of blobs in the sequence */
} StoreHeader;
\end{verbatim}

The \code{table_loc} member contains the file offset of a special blob
which stores the blob table.  The blob table contains an array of
\ctype{BlobDesc} structures.  Each \ctype{BlobDesc} describes one blob
in the file.  All blob descriptors are kept together to minimise the
amount of data which will be paged in while accessing the blob
meta-data.

As the blob table grows it will move around in the file.  The
\code{table_size} member records the number of array elements
allocated in the blob table while \code{table_len} records the length
of the blob table.  Once allocated a blob cannot be deleted, it can
only be marked as free space.

The \code{table_index} member records the blob table index of the blob
which contains the blob table.

The blob table entries are always arranged in strict ascending
sequence of the offset of the data that they describe.  This means
that as blobs are resized they may be handled by different entries in
the blob table.

To present external code with blob numbers which do not change after
the initial blob allocation, a blob sequence is maintained.  The blob
sequence is a simple indirection table which translates external blob
index into an internal blob table index.  This allows the table entry
for a blob to change without affecting external code.

The \code{seq_index} member records the blob table index of the blob
which describes the blob sequence.
\end{ctypedesc}

\begin{ctypedesc}[BlobDesc]{BlobDesc}
The blob table is constructed from an array of these structures.  Its
definition, found in \file{storage.h}, is:

\begin{verbatim}
typedef struct {
    off_t loc;                  /* location of the blob */
    size_t size;                /* size of blob */
    size_t len;                 /* length of blob */
    int status;                 /* status of blob */
    int type;                   /* user: type of blob */
    int other;                  /* user: index of related blob */
} BlobDesc;
\end{verbatim}

The \code{loc} member contains the file offset of the start of the
blob data.  The \code{size} member contains the allocated size of the
blob.  The \code{loc} member of the next \ctype{BlobDesc} \emph{is
always equal to} \code{loc + size} of this \ctype{BlobDesc}.  The
\code{len} member contains size of the blob requested by the external
code.

The \code{status} member records the status of this space controlled
by this blob.  The value will be one of:

\begin{longtable}{l|l}
BLOB_FREE & The blob is free space. \\
BLOB_TABLE & The blob contains the blob table. \\
BLOB_SEQUENCE & The blob contains the blob sequence. \\
BLOB_DATA & The blob contains user data. \\
\end{longtable}

The \code{type} and \code{other} members are not used by the Storage
API.  Applications are free to store whatever they wish in these
members.
\end{ctypedesc}

\begin{ctypedesc}[MmapBlobStore]{MmapBlobStore}
This structure is allocated by the \cfunction{store_open} function
when the blob storage is opened.  Its definition, found in
\file{storage.h}, is:

\begin{verbatim}
typedef struct {
    int mode;                   /* file open mode */
    int fd;                     /* file descriptor */
    int prot;                   /* mmap prot */
    StoreHeader *header;        /* address of start of store */
    size_t size;                /* size of file */
    int cycle;                  /* increment when file remapped */
} MmapBlobStore;
\end{verbatim}
\end{ctypedesc}

\begin{cfuncdesc}{MmapBlobStore *}{store_open}{char *filename, char *mode}
Opens the file specified in the \var{filename} argument using the mode
specified in the \var{mode} argument.

\begin{longtable}{l|l}
\code{'r'} & Open an existing file in read only mode. \\
\code{'r+'} & Open an existing file in read-write mode. \\
\code{'w+'} & Create a new file in read-write mode. \\
\end{longtable}

If successful the function allocates and returns a
\ctype{MmapBlobStore} structure which is passed as the first argument
to all other Storage API functions.  If the operation fails for any
reason the function will set a Python exception and will return
\NULL{}.
\end{cfuncdesc}

\begin{cfuncdesc}{int}{store_close}{MmapBlobStore *sm}
Closes the blob store and frees the \ctype{MmapBlobStore} structure.
On failure the function will set a Python exception and will return
\code{-1}.  On success the function returns \code{0}.
\end{cfuncdesc}

\begin{cfuncdesc}{int}{store_get_header}{MmapBlobStore *sm, StoreHeader *header}
Copies the contents of the header at the start of the memory mapped
file into the buffer supplied as the \var{header} argument.

The function returns \code{0} on success and \code{-1} on failure.
\end{cfuncdesc}

\begin{cfuncdesc}{void}{store_usage}{MmapBlobStore *sm, size_t *used, size_t *unused}
Traverses the blob table and returns the amount of space in use in the
\var{usage} argument and the amount of space which is either free or
wasted in the \var{unused} argument.
\end{cfuncdesc}

\begin{cfuncdesc}{size_t}{store_compress}{MmapBlobStore *sm}
Compresses the blob store to remove any space contained in free blobs
or at the end of existing blobs.  Note this function is not current
implemented.

Returns \code{0} on success and code{-1} on failure.
\end{cfuncdesc}

\begin{cfuncdesc}{int}{store_cycle}{MmapBlobStore *sm}
Every time the file is remapped due to size changes, the \code{cycle}
member of the \ctype{MmapBlobStore} is incremented.  This function
returns the current value of the \code{cycle} member.
\end{cfuncdesc}

\begin{cfuncdesc}{int}{store_num_blobs}{MmapBlobStore *sm}
Returns the number of blobs in the blob sequence.
\end{cfuncdesc}

\begin{cfuncdesc}{BlobDesc *}{store_get_blobdesc}{MmapBlobStore *sm, int index, int from_seq}
Returns a pointer to the blob table entry which describes the blob
specified by the \var{index} argument.  If the \var{from_seq} argument
is non-zero then the \var{index} argument will be used to look up the
blob sequence to obtain the index into the blob table.  When
\var{from_seq} is zero the index is used directly to index the blob
table.

If the \var{index} argument is outside the bounds of the blob sequence
or table the function will set a Python \exception{IndexError}
exception and will return \NULL{}.
\end{cfuncdesc}

\begin{cfuncdesc}{void *}{store_blob_address}{MmapBlobStore *sm, BlobDesc *desc}
Returns the address of the start of the data for the blob descriptor
passed in \var{desc}.  The \cfunction{store_blob_blobdesc()} returns
values suitable for use as the \var{desc} argument.

The function preforms no checking on the arguments.
\end{cfuncdesc}

\begin{cfuncdesc}{int}{store_append}{MmapBlobStore *sm}
Allocates a new entry in the blob sequence and returns the index.
Note that no blob data is allocated until
\cfunction{store_blob_resize()} is called to allocate some space for
the blob.

If the operation fails for any reason the function will set a Python
exception and will return \code{-1}.
\end{cfuncdesc}

\begin{cfuncdesc}{int}{store_blob_size}{MmapBlobStore *sm, int index}
Return the length of the blob identified by the blob sequence index in
the \var{index} argument.
\end{cfuncdesc}

\begin{cfuncdesc}{int}{store_blob_resize}{MmapBlobStore *sm, int index, size_t data_len}
Resize the blob identified by the blob sequence index in the
\var{index} argument to be the new size specified in the
\var{data_len} argument.

If you are growing the blob, the address of the blob will almost
certainly change after calling this function.  You must call
\cfunction{store_blob_blobdesc()} and \cfunction{store_blob_address()}
to obtain a valid address for the blob.
\end{cfuncdesc}

\begin{cfuncdesc}{int}{store_blob_free}{MmapBlobStore *sm, int index}
Free the blob identified by the blob sequence index in the
\var{index} argument.  The space will be reused.
\end{cfuncdesc}
