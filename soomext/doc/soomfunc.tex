%
%   The contents of this file are subject to the HACOS License Version 1.2
%   (the "License"); you may not use this file except in compliance with
%   the License.  Software distributed under the License is distributed
%   on an "AS IS" basis, WITHOUT WARRANTY OF ANY KIND, either express or
%   implied. See the LICENSE file for the specific language governing
%   rights and limitations under the License.  The Original Software
%   is "NetEpi Analysis". The Initial Developer of the Original
%   Software is the Health Administration Corporation, incorporated in
%   the State of New South Wales, Australia.
%
%   Copyright (C) 2004,2005 Health Administration Corporation.
%   All Rights Reserved.
%
\section{\module{soomfunc}}

\declaremodule{extension}{soomfunc}

The \module{soomfunc} module contains a collection of set operations
which can be performed on Numpy arrays:

The \module{soomfunc} module contains the following:

\begin{datadesc}{__version__}
A string which specifies the version of the \module{soomfunc} module.
\end{datadesc}

\begin{funcdesc}{unique}{a \optional{, b}}
Remove all duplicate values from the sorted rank-1 array passed as the
\var{a} argument returning a new rank-1 array of unique values.  If
the optional second argument \var{b} is supplied it will be used to
store the result.

\begin{verbatim}
>>> from soomfunc import * 
>>> from Numeric import *
>>> unique(array([1,2,3,3,4,4,5]))
array([1, 2, 3, 4, 5])
\end{verbatim}
\end{funcdesc}

\begin{funcdesc}{intersect}{a, b \optional{, \ldots}}
Return the unique intersection of the sorted rank-1 arrays passed.
All arrays must have the same typecode.

The function finds the smallest array then allocates an array of that
length to hold the result.  The result is primed by finding the
intersection of the smallest array with the first array in the
argument list (or the second if the first is the smallest).  The
result is then incrementally intersected in-place with the remainder
of the arrays generating a new result.

With each intersection, the function compares the length of the two
arrays.  If one array is at least three times larger than the other,
the \function{sparse_intersect()} method is used, otherwise the
\function{dense_intersect()} method is used.

\begin{verbatim}
>>> from soomfunc import * 
>>> from Numeric import *
>>> intersect(array([1,2,3]), array([2,3]), array([3,4]))
array([3])
\end{verbatim}
\end{funcdesc}

\begin{funcdesc}{sparse_intersect}{a, b \optional{, \ldots}}
Return the intersection of the sorted rank-1 arrays passed.  All
arrays must have the same typecode.  This generates the same result as
the \function{intersect()} function.

When finding the intersection of two arrays it uses a binary search to
locate matching values.  The first point for the binary search is
determined by dividing the number of elements remaining in the target
array by the number of elements remaining in the source array.

This method can be thousands of times faster than the
\function{dense_intersect()} function.  For arrays of similar size,
the \function{dense_intersect()} function will be more than twice as
fast as this function.

You are encouraged to use the \function{intersect()} function as it
will automatically use the best intersection function.
\end{funcdesc}

\begin{funcdesc}{dense_intersect}{a, b \optional{, \ldots}}
Return the intersection of the sorted rank-1 arrays passed.  All
arrays must have the same typecode.  This generates the same result as
the \function{intersect()} function.

When finding the intersection of two arrays it steps through both
arrays one value at a time to locate matching values.  The
\function{sparse_intersect()} function can be thousands of times
faster than this function for some inputs.

You are encouraged to use the \function{intersect()} function as it
will automatically use the best intersection function.
\end{funcdesc}

\begin{funcdesc}{outersect}{a, b \optional{, \ldots}}
Return the unique symmetric difference of the sorted rank-1 arrays
passed.  All arrays must have the same typecode.

Steps through all arrays in lock-step and finds all values which do
not occur in every array.  This is the exact opposite of the
\function{intersect()} function.

\begin{verbatim}
>>> from soomfunc import * 
>>> from Numeric import *
>>> outersect(array([1,2,3]), array([2,3]), array([3,4]))
array([1, 2, 4])
\end{verbatim}
\end{funcdesc}

\begin{funcdesc}{union}{a, b \optional{, \ldots}}
Return the unique union of the sorted rank-1 arrays passed.  All
arrays must have the same typecode.

Steps through all arrays in lock-step and finds all unique values
across every array.

\begin{verbatim}
>>> from soomfunc import * 
>>> from Numeric import *
>>> union(array([1,2,3,3]), array([2,2,3]), array([3,4]))
array([1, 2, 3, 4])
\end{verbatim}
\end{funcdesc}

\begin{funcdesc}{difference}{a, b \optional{, \ldots}}
Return the result of subtracting the second and subsequent sorted
rank-1 arrays from the first sorted rank-1 array.  All arrays must
have the same typecode.

\begin{verbatim}
>>> from soomfunc import * 
>>> from Numeric import *
>>> difference(array([1,2,3,3]), array([2,2,3]), array([3,4]))
array([1])
\end{verbatim}
\end{funcdesc}

\begin{funcdesc}{valuepos}{a, b}
Return an array of indexes into rank-1 array \var{a} where the values
in rank-1 array \var{b} appear.  Both arrays must have the same
typecode.

\begin{verbatim}
>>> from soomfunc import * 
>>> from Numeric import *
>>> valuepos(array([2,2,3,3,4,5,6]),array([3,5]))
array([2, 3, 5])
\end{verbatim}
\end{funcdesc}

\begin{funcdesc}{preload}{a \optional{, num \code{= -1}}}
Step backwards over \var{num} entries of the array in the \var{a}
argument forcing the pages into memory.  If \var{num} is less than
zero or greater than the length of the array, the entire array is
scanned.

This function is used to optimise the use of arrays stored in memory
mapped blobs.

\begin{verbatim}
>>> from soomfunc import * 
>>> from Numeric import *
>>> preload(array([2,2,3,3,4,5,6]))
\end{verbatim}
\end{funcdesc}
